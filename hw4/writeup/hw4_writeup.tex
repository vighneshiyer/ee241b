\documentclass[11pt]{article}

\usepackage{float}
\usepackage{hyperref}
\usepackage{fullpage}
\usepackage{verbatim}
\usepackage{moreverb}
\usepackage{graphicx}
\usepackage{parskip}
\usepackage{amsmath}
\usepackage{etoolbox}
\usepackage[toc,page]{appendix}
\graphicspath{{images/}}
\usepackage{gensymb}

\usepackage{minted}
\let\verbatiminput=\verbatimtabinput
\def\verbatimtabsize{4\relax}

\begin{document}
\title{EE 241B HW4 Writeup}

\author{Vighnesh Iyer}
\date{}
\maketitle

\tableofcontents

\section{Delay Replicas}
\begin{quote}
	Your design has a critical path of 20 FO4 inverter delays at the nominal supply voltage. Your friend has suggested to design a replica circuit consisting of N FO1 inverters (where N is approximately 40.
\end{quote}

\subsection{Replica Design}
\begin{quote}
	Design the actual replica made of N-1 identical stages with the Nth stage having an increased fanout to match the actual critical path. What is the fanout of the Nth stage?
\end{quote}

We begin by calculating the critical path delay in terms of minimally sized inverter delays $t_{inv}$ using the method of logical effort. We first define all the variables that are used for logical effort delay calculations.

\begin{align}
	f &= \text{ fanout of single stage} = C_{out} / C_{in} \nonumber \\
	F &= \text { total fanout} = C_{L,total} / C_{in} \nonumber \\
	\text{EF} &= \text { effective fanout} = \text{LE} f \nonumber \\
	p &= \text{ intrinsic delay (ratio of PMOS to NMOS capacitance)}; p_{inv} = \gamma \nonumber \\
	\text{LE} &= \text { logical effort} = (R_{eq,gate} C_{in,gate}) / (R_{eq,inv} C_{in,inv}); LE_{inv} = 1 \nonumber \\
	\text{b} &= \text{ branching factor} = \frac{C_{L,on-path} + C_{L,off-path}}{C_{L,on-path}} \nonumber \\
	\text{PE} &= \text{ path effort} = (\prod \text{LE}) (\prod b) F \nonumber \\
	t_{p,gate} &= t_{inv}(p + \text{LE}f) = t_{inv}(p + \text{EF}) \nonumber
\end{align}

Now we calculate the critical path delay.

\begin{align}
	t_{delay,inv} &= t_{inv}(p_{inv} + \text{LE} f) \nonumber \\
	t_{delay,inv} &= t_{inv}(\gamma + 4) \nonumber \\
	t_{delay,crit-path} &= t_{delay,inv}  \cdot N_{stages} = 20 \cdot t_{inv}(\gamma + 4) \nonumber
\end{align}

Let's now consider the path that consists of 39 FO1 inverters and 1 larger inverter. We can break this path's delay into 2 sub-delays.

\begin{align}
	t_{delay,replica1} &= 39 \cdot t_{inv}(\gamma + 1) \nonumber \\
	t_{delay,replaca2} &= t_{inv}(\gamma + f_{last,stage}) \nonumber \\
	t_{delay,replica} &= 39 t_{inv}(\gamma + 1) + t_{inv}(\gamma + f_{last,stage}) \nonumber \\
	\text{Set } t_{delay,replica} &= t_{delay,crit-path} \nonumber 
\end{align}

\newpage
\appendix

\end{document}