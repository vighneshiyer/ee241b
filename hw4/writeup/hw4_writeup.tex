\documentclass[11pt]{article}

\usepackage{float}
\usepackage{hyperref}
\usepackage{fullpage}
\usepackage{verbatim}
\usepackage{moreverb}
\usepackage{graphicx}
\usepackage{parskip}
\usepackage{amsmath}
\usepackage{etoolbox}
\usepackage[toc,page]{appendix}
\graphicspath{{images/}}
\usepackage{gensymb}

\usepackage{minted}
\let\verbatiminput=\verbatimtabinput
\def\verbatimtabsize{4\relax}

\begin{document}
\title{EE 241B HW4 Writeup}

\author{Vighnesh Iyer}
\date{}
\maketitle

\tableofcontents

\section{Delay Replicas}
\begin{quote}
	Your design has a critical path of 20 FO4 inverter delays at the nominal supply voltage. Your friend has suggested to design a replica circuit consisting of N FO1 inverters (where N is approximately 40.
\end{quote}

\subsection{Replica Design}
\begin{quote}
	Design the actual replica made of N-1 identical stages with the Nth stage having an increased fanout to match the actual critical path. What is the fanout of the Nth stage?
\end{quote}

We begin by calculating the critical path delay in terms of minimally sized inverter delays $t_{inv}$ using the method of logical effort. We first define all the variables that are used for logical effort delay calculations.

\begin{align}
	f &= \text{ fanout of single stage} = C_{out} / C_{in} \nonumber \\
	F &= \text { total fanout} = C_{L,total} / C_{in} \nonumber \\
	\text{EF} &= \text { effective fanout} = \text{LE} f \nonumber \\
	p &= \text{ intrinsic delay (ratio of PMOS to NMOS capacitance)}; p_{inv} = \gamma \nonumber \\
	\text{LE} &= \text { logical effort} = (R_{eq,gate} C_{in,gate}) / (R_{eq,inv} C_{in,inv}); LE_{inv} = 1 \nonumber \\
	\text{b} &= \text{ branching factor} = \frac{C_{L,on-path} + C_{L,off-path}}{C_{L,on-path}} \nonumber \\
	\text{PE} &= \text{ path effort} = (\prod \text{LE}) (\prod b) F \nonumber \\
	t_{p,gate} &= t_{inv}(p + \text{LE}f) = t_{inv}(p + \text{EF}) \nonumber
\end{align}

Now we calculate the critical path delay.

\begin{align}
	t_{delay,inv} &= t_{inv}(p_{inv} + \text{LE} f) \nonumber \\
	t_{delay,inv} &= t_{inv}(\gamma + 4) \nonumber \\
	t_{delay,crit-path} &= t_{delay,inv}  \cdot N_{stages} = 20 \cdot t_{inv}(\gamma + 4) \nonumber
\end{align}

Let's now consider the path that consists of 39 FO1 inverters and 1 larger inverter. We can break this path's delay into 2 sub-delays.

\begin{align}
	t_{delay,replica1} &= 39 \cdot t_{inv}(\gamma + 1) \nonumber \\
	t_{delay,replaca2} &= t_{inv}(\gamma + f_{last,stage}) \nonumber \\
	t_{delay,replica} &= 39 t_{inv}(\gamma + 1) + t_{inv}(\gamma + f_{last,stage}) \nonumber \\
	\text{Set } t_{delay,replica} &= t_{delay,crit-path} \nonumber 
\end{align}

Performing calculations, and assuming $\gamma = 1$ for a modern process, we get $f_{last,stage}$ = 21, and an approximate area reduction for the replica from 80 to 60.

\subsection{Replica Tracking Across Different Gates}
\begin{quote}
	There are three nearly identical critical paths in the circuit, one consisting of 20 FO4 inverters, one consisting of all NAND2s and one consisting of all NOR2s. Simulate tracking of the replica from part a) from the nominal down to 0.5V with 50 mV steps.
\end{quote}

\section{Power-performance Tradeoffs}

\subsection{Energy/Delay Sensitivity}
\begin{quote}
	Using the alpha-power law model, derive the analytical expression for the energy/delay sensitivity of a design to scaling of supply voltage. Evaluate this expression at the nominal supply voltage, using the results from Homework 1 ($K = 0.001, \alpha = 1.5, V_{th} = 0.45 \text{V for NMOS}$)
\end{quote}

We can model the switching NMOS transistor as a current source (dis)charging its self-capacitance and load capacitance which we lump together as $C$.

\begin{align}
	I_D &= K (V_{GS} - V_{th})^\alpha \propto t_{delay} \nonumber \\
	V_{GS} &= V_{DD} \text{ assuming instant switching} \nonumber \\
	\text{for NMOS's capacitor } V(t) &= \frac{Q(t)}{C} = \frac{1}{C}\int_{0}^{t}I(t) dt = \frac{1}{C}\int_{0}^{t} K (V_{DD} - V_{th})^\alpha dt \nonumber \\
	&= \frac{1}{C} K (V_{DD} - V_{th})^\alpha t = V_{DD} \nonumber \\
	t_{delay} &= \frac{V_{DD} K C}{(V_{DD} - V_{th})^\alpha} \nonumber \\
	E_{gate} &= C V_{DD}^2 \nonumber
\end{align}

To get sensitivities to $V_{DD}$ take derivatives with respect to $V_{DD}$.

\begin{align}
	\frac{\partial t}{\partial V_{DD}} = K C \frac{(\alpha - 1)V_{DD} + V_{th}}{-(V_{DD} - V_{th})^{\alpha + 1}} \nonumber \\
	\frac{\partial E}{\partial V_{DD}} = 2 C V_{DD} \nonumber \\
	\frac{\partial E}{\partial V_{DD}} \bigg/ \frac{\partial t}{\partial V_{DD}} = \frac{2 V_{DD}}{K} \cdot \frac{-(V_{DD} - V_{th})^{\alpha+1}}{(\alpha - 1)V_{DD} + V_{th}} \nonumber 
\end{align}

In this calculation we don't factor leakage energy into the gate's energy since we are concerned with energy vs delay tradeoffs, which involve switching energy.

\subsection{Energy/Delay Sensitivity to Logic Depth}
\begin{quote}
	What is the energy/delay sensitivity of a design to the logic depth? You can define the logic depth as a number of FO4 inverters that fit into one clock period.
\end{quote}

Lets say that there are $N$ inverter stages in a logic path. Then, the delay is linearly proportional to $N$ because the fanout of each stage is fixed and each inverter is sized to have the same pull-up/pull-down strength as the minimally sized inverter.

\begin{align}
	t_{delay,path} \propto N \cdot t_{delay} = \frac{V_{DD} K C N}{(V_{DD} - V_{th})^\alpha} \nonumber 
\end{align}

However, the energy is dependent on the total capacitance of each inverter along the path:

\begin{align}
	E_{path} = V_{DD}^2 \cdot (C + \sum_{x = 1}^{N}(C \cdot 4^x)) = V_{DD}^2 (C + \frac{4}{3} C (4^N - 1)) \nonumber
\end{align}

Taking derivatives wrt $N$:

\begin{align*}
	\frac{\partial t_{delay,path}}{\partial N} = \frac{V_{DD} K C}{(V_{DD} - V_{th})^\alpha} \\
	\frac{\partial E_{path}}{\partial N} = \frac{1}{3} V_{DD}^2 \cdot C \cdot 4^{N+1} \ln(4) \\
	\frac{\partial E_{path}}{\partial N} \bigg/ \frac{\partial t_{delay,path}}{\partial N} = \frac{V_{DD} 4^{N+1} \ln(4) (V_{DD} - V_{th})^\alpha}{3K}
\end{align*}

So it can be concluded that the energy is much more sensitive to changes in the logic depth than the delay as would be expected.

\section{Subthreshold Design}
\subsection{Subthrehold Dependence on Supply Voltage}
\begin{quote}
	From the lecture notes, it appears that for a design operating in subthreshold, the minimum energy point of a design does not depend on the threshold voltage, and only depends on the supply. Can you explain why (very briefly)?
\end{quote}

To be brief, the leakage current becomes active switching current when operating in subthreshold. Then the leakage energy becomes dependent not only on the leakage current but the delay, which is affected in the opposite way by the threshold voltage. These two terms 'cancel out' and the threshold voltage doesn't make a difference in the minimum energy point.

\subsection{Threshold Voltage Importance}
\begin{quote}
	If it doesn't matter for the energy consumption, why is it important to control the transistor threshold voltage in subthreshold design?
\end{quote}

Delay is very sensitive to the threshold voltage when $V_{DD}$ is near the threshold voltage. So even though energy savings don't depend on the threshold, the circuit timing and retention-ability of static circuits depend greatly on the threshold voltage.

\section{Stack Forcing}
\begin{quote}
	What is the sensitivity in the energy-delay space of stack forcing technique, applied to every inverter in an optimally sized (for minimum delay) inverter chain of length 6? The input capacitance is 1fF, and the initial leakage is 30\% of the total energy dissipated by the inverters. Assume that the $C_{d} = C_{g} = 2$ fF$/\mu$, and a linear delay model is used. $V_{DD}$ = 1V and the stack of 2 reduces the leakage by a factor of 10.
\end{quote}

Taking the original 6 inverter chain, we want to find its delay and energy consumption. The fanout of each inverter ought to be equal to minimize delay, and we can simply assume a fanout of 4 for each inverter (although this would be process dependent and based on $\gamma$; we assume $\gamma = 1$ in this problem just to make calculations easy). We also assume that the PMOS mobility is half that of the NMOS.

The minimally sized inverter is to have 1fF of input capacitance so it stands that the NMOS is 1/6 $\mu$ and the PMOS is 2/6 $\mu$ in width.

\begin{align}
	t_{inv,minimal} &= \ln(2) R \text{ 1 fF} \nonumber \\
	t_{inv,FO4} &= t_{inv,minimal} (\gamma + f) = t_{inv,minimal} \cdot 5 = \ln(2) R \text{ 5 fF} \nonumber \\
	t_{path} &= 6 \cdot t_{inv,FO4} = \ln(2) R \text{ 30 fF} \nonumber \\
	E_{path,dyn} &= C_{total} V_{DD}^2 = (1 + \sum_{x=1}^{5}2 \cdot 4^x + 4096) 1^2 = (1 + 2728 + 4096) = 6825 \text{ fJ} \nonumber \\
	E_{path,leak} &= E_{path,dyn} \cdot \frac{0.3}{0.7} = 2925 \text{ fJ}\nonumber \\
	E_{path,total} &= 9570 \text{ fJ} \nonumber
\end{align}

Now we assume that stack forcing is on and that every inverter is made up of 2 stacked NMOSs and 2 stacked PMOSs with their gates tied together as usual. We also assume that the widths of the NMOS and PMOS transistors are cut in half to maintain the same input capacitance and fanout. This results in an effective switching resistance that is 4x greater than the un-stacked inverter due to the halving of widths and the stacking (assuming linear delay model). The self-loading capacitance of each inverter goes down by 2x assuming we neglect the stack node capacitance.

\begin{align}
	%t_{inv,minimal} &= \ln(2) 4 R \text{ 0.5 fF} \nonumber \\
	t_{inv,FO4} &= \ln(2) 4R \text{ 4.5 fF} \nonumber \\
	t_{path} &= 6 \cdot \ln(2) 4R \text{ 4.5 fF} \nonumber \\
	E_{path,dyn} &= (.5 + \sum_{x=1}^{5}(4^x + 4^x/2) + 4096) 1^2 = 6142.5 \text{ fJ} \nonumber \\
	E_{path,leak} &= E_{path,leak,orig} / 10 = 292.5 \text{ fJ} \nonumber \\
	E_{path,total} &= 6435 \text{ fJ} \nonumber
\end{align}

We can now calculate the sensitivity:

\begin{align}
	\frac{\partial E}{\partial \text{stack}} \bigg/ \frac{\partial t}{\partial \text{stack}} = \frac{\Delta E}{\Delta t} &= \frac{9570 - 6435 \text{ fJ}}{\ln(2) R \text{ 30 fF} - \ln(2) R \text{ 108 fF}} \nonumber \\
	&= \frac{3135 \text{ fJ}}{-R \cdot 54 \text{ fF}} \nonumber
\end{align}

where $R$ can be approximated through simulating the switching current characteristic of an inverter.

\newpage
\appendix

\end{document}